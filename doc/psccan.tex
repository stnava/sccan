\documentclass{llncs}
\newcommand{\Section}[1]{\vspace{-8pt}\section{\hskip-1em.~~#1}\vspace{-3pt}}
\newcommand{\SubSection}[1]{\vspace{-3pt}\subsection{\hskip -1em.~~#1}\vspace{-3pt}}
\usepackage{amsmath,amssymb}
\usepackage{times}
\usepackage{setspace,verbatim}
\usepackage{epsfig}

\begin{document}
\vspace{-0.1in}
\title{Partial sparse canonical correlation analysis for population
  studies in neuroimaging}
\author{Anonymous}
\institute{Anonymous}
\maketitle              
%\vspace{-0.1in}
\begin{abstract}
Sparse canonical correlation analysis (SCCA) is a powerful,
multivariate statistical tool for making unbiased inferences about the
relationship between different types of measurements taken on the same
population.  Previously, SCCA depended upon univariate models to
factor out the effects of confounding variables.  Here, we present a
new algorithm for computing partial sparse canonical correlation
analysis (PSCCA).  PSCCA employs three views of the data with one
treated as a special set of measurements whose effect on the other two
should be factored out.  For instance, the relationship between
cognition and neuroanatomy should be measured independently of the
effect of age and gender.  This work gives a concise summary of the
theory and algorithms underlying PSCCA along with a brief example
application taken from neuroimaging.
\end{abstract}
%\vspace{-0.2in}
\section{Introduction}
% pubmed references to MRI :  
% 2000-2001 --- 25561
% 2001-2002 --- 27053
% 2003-2004 --- 31708
% 2005-2006 --- 38620
% 2008-2009 --- 49288
% 2009-2010 --- 49323
% pubmed references to MRI brain :  
% 2000-2001 --- 9938
% 2001-2002 --- 
% 2003-2004 --- 12655
% 2005-2006 --- 
% 2008-2009 --- 
% 2009-2010 --- 19676
The number of neuroimaging studies published annually doubled from
9938 in 2000-2001 to 19676 in 2009-2010
(http://www.ncbi.nlm.nih.gov/pubmed/).  This growth has been
accompanied by increasing diversity in the types of data being
collected such that neuroimaging studies include not only various
structural and functional modalities but also neurocognitive
batteries, genetics and environmental measurements.  At the same
time, the statistical methods used in neuroimaging have changed
relatively little from twenty years ago with the exception of a small
number of recent studies \cite{Tosun2010a}.

Why SCCA

Why PSCCA

What we are doing here ( a few sentences )

%\vspace{-0.2in}
\section{PSCCA Theory}
SCCA will maximize:
\begin{equation}
\text{argmax}( \omega_T,\omega_{J}) :
% \omega_T {\bf T}^T {\bf J} \omega_{J}  - ( \lambda_T \| \omega_T \|_1 + \lambda_F \|  \omega_{J}  \|_1 ),
 ~\text{Corr}~(\underbrace{  {\bf T} ~\omega_T~}_{\text{Thickness Projections}} ,~\underbrace{ {\bf J} ~\omega_{J}~}_{\text{Jacobian Projections}}~) - \lambda_T \| \omega_T \|_1 - \lambda_J \|  \omega_{J}  \|_1 , 
\end{equation} 
where ${\bf T}$ is a matrix with columns containing voxels from the $\beta_T$ images and $n$-subjects number of rows, 
where ${\bf J}$ is a matrix with columns containing voxels from the $\beta_J$ images and $n$-subjects number of rows, 
 Corr computes Pearson correlation and the
$\lambda$ are inversely related to the sparseness costs, $C$.  %\vspace{-0.2in}

{\bf notes:}

PSCCA will not apply sparseness to the covariate view.  

Follow Timm's notation.  Let's use {\bf  X}, {\bf Y} and {\bf Z} to describe the theory, with {\bf Z} as the
'covariate'.   

\section{PSCCA Algorithm}
Use the Lanczos algorithm and soft-max.

\section{Results}
Apply to one of our imaging datasets, potentially Phil's OCT data.

\section{Discussion}
%\vspace{-0.1in}

\noindent{\bf Acknowledgment}
 This work is supported by Grant XXX 
% 1R01EB006266-01 
% from the ...
%National Institute Of Biomedical Imaging and Bioengineering and administered through the UCLA Center for Computational Biology.
% %\vspace{-0.1in}
\bibliographystyle{IEEEbib}
\bibliography{./cca.bib}

\end{document}
