\documentclass{llncs}
\newcommand{\Section}[1]{\vspace{-8pt}\section{\hskip-1em.~~#1}\vspace{-3pt}}
\newcommand{\SubSection}[1]{\vspace{-3pt}\subsection{\hskip -1em.~~#1}\vspace{-3pt}}
\usepackage{amsmath,amssymb}
\usepackage{times}
\usepackage{setspace,verbatim}
\usepackage{epsfig}

\begin{document}
\vspace{-0.1in}
\title{Partial sparse canonical correlation analysis for population
  studies in neuroimaging}
\author{Anonymous}
\institute{Anonymous}
\maketitle              
%\vspace{-0.1in}
\begin{abstract}
Sparse canonical correlation analysis (SCCA) is a powerful,
multivariate statistical tool for making unbiased inferences about the
relationship between different types of measurements taken on the same
population.  Previously, SCCA depended upon univariate models to
factor out the effects of confounding variables.  Here, we present a
new algorithm for computing partial sparse canonical correlation
analysis (PSCCA).  PSCCA employs three views of the data with one
treated as a special set of measurements whose effect on the other two
should be factored out.  For instance, the relationship between
cognition and neuroanatomy should be measured independently of the
effect of age and gender.  This work gives a concise summary of the
theory and algorithms underlying PSCCA along with a brief example
application taken from neuroimaging.
\end{abstract}
%\vspace{-0.2in}
\section{Introduction}
% pubmed references to MRI :  
% 2000-2001 --- 25561
% 2001-2002 --- 27053
% 2003-2004 --- 31708
% 2005-2006 --- 38620
% 2008-2009 --- 49288
% 2009-2010 --- 49323
% pubmed references to MRI brain :  
% 2000-2001 --- 9938
% 2001-2002 --- 
% 2003-2004 --- 12655
% 2005-2006 --- 
% 2008-2009 --- 
% 2009-2010 --- 19676
The number of neuroimaging studies published annually doubled from
9938 in 2000-2001 to 19676 in 2009-2010
(http://www.ncbi.nlm.nih.gov/pubmed/).  This growth has been
accompanied by increasing diversity in the types of data being
collected such that neuroimaging studies include not only various
structural and functional modalities but also neurocognitive
batteries, genetics and environmental measurements.  At the same
time, the statistical methods used in neuroimaging have changed
relatively little from twenty years ago with the exception of a small
number of recent studies \cite{Tosun2010a}.

Why SCCA

Why PSCCA

for instance, consider a standard neuroimaging population study.  each
subject in the study is between 20 and 40 years of age, has a
quantitative imaging measurement, is a member of one of two groups
(patient or control) and is either male or female.  a standard
regression analysis will treat the quantitative imaging measurement as the dependent variable and the
age, gender and diagnosis group as predictors where each test is
performed independently at each voxel.  detection power is
compromised by the multiple comparisons problem incurred by the huge
number of imaging measurements (millions) as well as the confounding
variables (age, gender).  the pscca version of the
problem is formulated similarly.  however, the test itself is
performed globally via a sparse multivariate formulation over the
whole region of interest (e.g. all voxels in the cortex).  thus, pscca
uses only a single test on the full set of data.   the sparseness
parameter in pscca selects the subest of the brain most associated
with the variable of interest (diagnosis) while factoring out
confounding effects (age, gender).  

What we are doing here ( a few sentences )

this paper will detail and illustrate our approach to performing
neuroimaging studies in the style of traditional formulations but
using a new, powerful multivariate pscca.  we highlight in both
simulated and real data the advantages and disadvantages of this
method.

%\vspace{-0.2in}
\section{PSCCA Theory}
SCCA will maximize:
\begin{equation}
\text{argmax}( \omega_T,\omega_{J}) :
% \omega_T {\bf T}^T {\bf J} \omega_{J}  - ( \lambda_T \| \omega_T \|_1 + \lambda_F \|  \omega_{J}  \|_1 ),
 ~\text{Corr}~(\underbrace{  {\bf T} ~\omega_T~}_{\text{Thickness Projections}} ,~\underbrace{ {\bf J} ~\omega_{J}~}_{\text{Jacobian Projections}}~) - \lambda_T \| \omega_T \|_1 - \lambda_J \|  \omega_{J}  \|_1 , 
\end{equation} 
where ${\bf T}$ is a matrix with columns containing voxels from the $\beta_T$ images and $n$-subjects number of rows, 
where ${\bf J}$ is a matrix with columns containing voxels from the $\beta_J$ images and $n$-subjects number of rows, 
 Corr computes Pearson correlation and the
$\lambda$ are inversely related to the sparseness costs, $C$.  %\vspace{-0.2in}

{\bf notes:}

PSCCA will not apply sparseness to the covariate view.  

Follow Timm's notation.  Let's use {\bf  X}, {\bf Y} and {\bf Z} to describe the theory, with {\bf Z} as the
'covariate'.   

\section{PSCCA Algorithm}
Use the Lanczos algorithm and soft-max.

\section{Results}
\subsection{synthetic data}
we simulate three view data with and without a significant association
between the primary and secondary variables.  this analysis can be
achieved by simulating imaging data and two other views (age,
cognition) in such a way that the age is the true hidden variable that
generates both cognition and imaging measurements.  pscca should then
detect an insignificant association between cognition and imaging when
age is used as the confounding variable.  similarly, pscca should
detect a significant association between age and imaging when
cognition is a confounding variable.   

\subsection{real data versus regression}
Apply to one of our imaging datasets, potentially Phil's OCT data.
Alternatively, the oasis data. 

We employ a subset of the freely available OASIS dataset and compare
regression and pscca-based detection of gray matter effects
attributable to dementia.

PSCCA shows significant effects using a global multivariate test at
the $p<0.05$ level, permutation tested.  

Regression does not reveal signficant effects at the FDR-corrected
$q<0.05$ level.  

Selected slices of the results compare the regression-based p-value
map to the pscca weight vector map.  

we use only the primary eigenvector from pscca.  future work will
analyze the effect of including additional eigenvectors. 

\section{Discussion}
%\vspace{-0.1in}

\noindent{\bf Acknowledgment}
 This work is supported by Grant XXX 
% 1R01EB006266-01 
% from the ...
%National Institute Of Biomedical Imaging and Bioengineering and administered through the UCLA Center for Computational Biology.
% %\vspace{-0.1in}
\bibliographystyle{IEEEbib}
\bibliography{./cca.bib}

\end{document}
